\documentclass[12pt,a4paper]{article}
\usepackage[english]{babel}
\usepackage{amsmath}
\usepackage{amsfonts}
\usepackage{amssymb}
\usepackage{blindtext}
\usepackage[left=2.54cm,right=2.54cm,top=2.54cm,bottom=2.54cm]{geometry}
\usepackage{ragged2e} 
\usepackage{setspace}
\usepackage{array}
\usepackage{caption}
\usepackage{subcaption}
\usepackage{longtable}
\usepackage{graphicx}
\usepackage{enumitem}
\usepackage{multirow}
\usepackage[numbers]{natbib}
\usepackage{lscape} % Optional if table is too wide
%\usepackage{hyperref}
%\usepackage{lineno}
\usepackage{float}

\captionsetup{
	labelfont=bf, 
	textfont=it, 
	labelsep=colon, 
	aboveskip=5pt, 
	belowskip=5pt  
}

\newcolumntype{C}[1]{>{\centering\arraybackslash}m{#1}} 
\newcolumntype{L}[1]{>{\raggedright\arraybackslash}m{#1}}
\newcolumntype{R}[1]{>{\raggedleft\arraybackslash}m{#1}} 
\newcolumntype{J}[1]{>{\arraybackslash}m{#1}} 
\renewcommand{\baselinestretch}{1.15}
\setlength{\parindent}{0pt}
\setlength{\parskip}{1em} 

\begin{document}
	%\linenumbers
	
	\begin{justify}
		\textbf{\Large
			Machine Learning Estimation of undrained shear strength of Bangkok clay [Tentative] 
		}
	\end{justify}
	
	\begin{flushleft}
		\textbf{Viroon Kamchoom  \textsuperscript{a}, Ankit Garg\textsuperscript{b}, Sai krishna Akash Ramineni \textsuperscript{c}, Thanu Harnpattanapanich \textsuperscript{d,e}, Phichet Ratanaprasatkul \textsuperscript{f}}
	\end{flushleft}
	
	\begin{flushleft}
		{\tiny {\textsuperscript{a}Excellent centre for green and sustainable infrastructure, Department of Civil Engineering, School of Engineering, King Mongkut's Institute of Technology Ladkrabang (KMITL), Bangkok, Thailand \\ 
				\textsuperscript{b}University in Shantou, Shantou, China \\ 
				\textsuperscript{c}Undergraduate, Department of Civil Engineering at IIT Indore, Indore city, Madhya Pradesh, Postal code- 453552, India \\
				\textsuperscript{d} Geotech Pillar co., ltd, Bangkok, Thailand \\
				\textsuperscript{e} Thailand Underground and Tunnelling Group, The Engineering Institute of Thailand, Thailand \\
				\textsuperscript{f} Bureau of Engineering and Architectural Design, Royal Irrigation Department (RID), Bangkok, Thailand \\
		}}
	\end{flushleft}
	
	\begin{abstract}
		content...
	\end{abstract}
	
	\hspace{0.4cm} \textbf{Keywords:} 
	
	\section{Introduction}
	
	\section{Material and Methods}
	
	\subsection{Soil sample collection and testing}
	
	Soft Bangkok clay samples were collected from two different areas in the western part of Bangkok, Thailand. The soil investigation was carried out in order to obtain the design parameters for the water drainage tunnel under Bang Nam Chued and Likij canals, which are part of a national scheme to improve the drainage system and mitigate the recurring inundation problem in the lower Chao Phraya River Basin. A total of 54 boreholes were used in this study. Boreholes with a diameter of 10 cm were drilled using a power auger to a depth of 2–3 meters, and then percussion wash boring was carried out throughout the borehole depth. The soft clay layer was found to vary within 12–15 m depth. The undisturbed samples were collected according to ASTM D1587 by using a Thin-Walled Shelby Tube with a 7.5-cm diameter and a 75-cm length. The samples were collected in soft to medium clay layers at every 1 m. The tube was pressed into the soil about 0.5 m depth and then twisted to take each soil sample out of the borehole. After that, each tube was waxed on both sides to prevent any moisture loss from the soil and transported to the laboratory.
	
	The laboratory tests were carried out in accordance with ASTM standards. Bulk unit weight can be calculated from the ratio between the weight of the soil sample and the volume of the soil being measured. The natural water content was calculated based on ASTM D2216, which is the ratio between the weight of soil moisture and the weight of dry soil. The Atterberg limits test according to ASTM D4318 is used to determine the soil index properties that depend on the soil composition, such as mineral compounds in the soil, soil consistency, etc. Grain size distribution was obtained based on ASTM D422. The soil sample was oven-dried at approximately 105 °C for 24 h, after which the dried soil was sieved through a sieve number of 200 (for particles smaller than 75 microns). The hydrometer test according to ASTM D2487 was conducted for soil classification. The limits, such as LL (Liquid limit), PL (Plastic limit), and PI (Plastic Index), together with the grain size distribution, were used to determine the classification of fine-grained soils (i.e., ASTM D2487). The Unconfined Compression Test, referred to as ASTM D21466, was conducted to determine the undrained shear strength of the soil.
	
	\section{Results and Discussion}
	
	
	
	\section{Conclusion}
	
	\section*{Acknowledgments}
	
	The first author (V. Kamchoom) would like to thank the grant under Climate Change and Climate Variability Research in Monsoon Asia (CMONA) from National Research Council of Thailand (NRCT) and the National Natural Science Foundation of China (NSFC). The second author (A. Garg) thanks the grant XXX provided by XXX. 
	
	\bibliographystyle{unsrtnat}
	\bibliography{reference.bib}
	
\end{document}